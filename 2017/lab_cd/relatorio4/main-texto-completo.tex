\documentclass[
    % -- opções da classe memoir --
    12pt,               % tamanho da fonte
    openright,          % capítulos começam em pág ímpar (insere página vazia caso preciso)
    %twoside,            % para impressão em verso e anverso. Oposto a oneside
    oneside,
    a4paper,            % tamanho do papel.
    %chapter=TITLE,     % títulos de capítulos convertidos em letras maiúsculas
    %section=TITLE,     % títulos de seções convertidos em letras maiúsculas
    %subsection=TITLE,  % títulos de subseções convertidos em letras maiúsculas
    %subsubsection=TITLE,% títulos de subsubseções convertidos em letras maiúsculas
    % -- opções do pacote babel --
    english,            % idioma adicional para hifenização
    brazil           % o último idioma é o principal do documento
    ]{ufscar-sc}

\usepackage{graphicx} % Imagem com subimagens
\usepackage{subcaption} % Imagem com subimagens
\usepackage{float} % Forçar posicionamento das imagens
\usepackage{listings} % Inserção de códigos

% ---
% Pacotes básicos
% ---
\usepackage{lmodern}            % Usa a fonte Latin Modern
\usepackage[T1]{fontenc}        % Selecao de codigos de fonte.
\usepackage[utf8]{inputenc}     % Codificacao do documento (conversão automática dos acentos)
% ---

% ---
% Pacotes adicionais, usados apenas no âmbito do Modelo Canônico do abnteX2
% ---
\usepackage{lipsum}             % para geração de dummy text
\usepackage{blindtext}          % para geração de dummy text

% ---

% ---
% CONFIGURAÇÕES DE PACOTES ADICIONAIS UTEIS
% ---
\usepackage{pdfpages}			% para incluir arquivos pdf no documento
%\usepackage[final,obeyFinal]{todonotes}			% para não imprimir pendencias na versão final

\usepackage[printonlyused]{acronym}		% siglas


% ---
% Informações de dados para CAPA e FOLHA DE ROSTO
% ---
\titulo{Experimento 04 - Implementação de um completo com 4 bits utilizando Verilog}

% Trabalho individual
%\autor{AUTOR DO TRABALHO}

% Trabalho em Equipe
% ver também https://github.com/abntex/abntex2/wiki/FAQ#como-adicionar-mais-de-um-autor-ao-meu-projeto
\renewcommand{\imprimirautor}{
\begin{tabular}{lr}
Lorhan Sohaky de Oliveira Duda Kondo & 740951 \\
\end{tabular}
}


\departamento{Departamento de Computação}
\curso{Ciência da Computação}
\disciplina{Laboratório de Circuitos Digitais}

\preambulo{}

\local{São Carlos - SP}
\data{\the\year}

% Definir o que for necessário e comentar o que não for necessário
% Utilizar o Nome Completo
\orientador{Fredy João Valente}

\instituicao{%
  \imprimirifsp
  \par
  \imprimirdepartamento
  \par
  \imprimircurso
  \par
  \imprimirdisciplina
}

% ---


% ---
% Configurações de aparência do PDF final

% alterando o aspecto da cor azul
\definecolor{blue}{RGB}{41,5,195}

% informações do PDF
\makeatletter
\hypersetup{
        %pagebackref=true,
        pdftitle={\@title},
        pdfauthor={\@author},
        pdfsubject={\imprimirpreambulo},
        pdfcreator={LaTeX with abnTeX2},
        pdfkeywords={abnt}{latex}{abntex}{abntex2}{trabalho acadêmico},
        colorlinks=true,            % false: boxed links; true: colored links
        linkcolor=blue,             % color of internal links
        citecolor=blue,             % color of links to bibliography
        filecolor=magenta,              % color of file links
        urlcolor=blue,
        bookmarksdepth=4
}
\makeatother
% ---

% ---
% Espaçamentos entre linhas e parágrafos
% ---

% O tamanho do parágrafo é dado por:
\setlength{\parindent}{1.3cm}

% Controle do espaçamento entre um parágrafo e outro:
\setlength{\parskip}{0.2cm}  % tente também \onelineskip

% ---
% compila o indice
% ---
\makeindex
% ---


% ----
% Início do documento
% ----

\lstset{backgroundcolor=\color{white},   % choose the background color; you must add \usepackage{color} or \usepackage{xcolor}; should come as last argument
  basicstyle=\footnotesize,        % the size of the fonts that are used for the code
  breakatwhitespace=false,         % sets if automatic breaks should only happen at whitespace
  breaklines=true,                 % sets automatic line breaking
  captionpos=b,                    % sets the caption-position to bottom
  escapeinside={\%*}{*)},          % if you want to add LaTeX within your code
  extendedchars=true,              % lets you use non-ASCII characters; for 8-bits encodings only, does not work with UTF-8
  frame=single,	                   % adds a frame around the code
  keepspaces=true,                 % keeps spaces in text, useful for keeping indentation of code (possibly needs columns=flexible)
  keywordstyle=\color{blue},       % keyword style
  language=Verilog,                 % the language of the code
  rulecolor=\color{black},         % if not set, the frame-color may be changed on line-breaks within not-black text (e.g. comments (green here))
  showspaces=false,                % show spaces everywhere adding particular underscores; it overrides 'showstringspaces'
  showstringspaces=false,          % underline spaces within strings only
  showtabs=false,                  % show tabs within strings adding particular underscores
  stepnumber=2,                    % the step between two line-numbers. If it's 1, each line will be numbered
  stringstyle=\color{mymauve},     % string literal style
  tabsize=2,	                   % sets default tabsize to 2 spaces
  title=\lstname  }

\begin{document}

% Retira espaço extra obsoleto entre as frases.
\frenchspacing

\newpage

% ----------------------------------------------------------
% ELEMENTOS PRÉ-TEXTUAIS
% ----------------------------------------------------------
\pretextual

% ---
% Capa
% ---
\imprimircapa

% ---
% Folha de rosto
% (o * indica que haverá a ficha bibliográfica)
% ---
\imprimirfolhaderosto
%\imprimirfolhaderosto*
% ---


% ---
% inserir lista de ilustrações
% ---
\pdfbookmark[0]{\listfigurename}{lof}
\listoffigures*
\cleardoublepage
% ---

% ---
% inserir lista de tabelas
% ---
\pdfbookmark[0]{\listtablename}{lot}
\listoftables*
\cleardoublepage
% ---

% ---
% inserir lista de quadros
% ---
%\pdfbookmark[0]{\listofquadrosname}{loq}
%\listofquadros*
%\cleardoublepage
% ---

% ---
% inserir lista de abreviaturas e siglas
% ---

% Utilizando acronym para tratar a apresentação de siglas e abntex para fazer a lista
% \item para abntex
% \acrodef para acronym
% o padrão da classe acronym é somente listas as siglas em uso, já o abntex vai listar todas definidas.

% a lista deve ser definida em ordem alfabética pois o abntex não ordena automaticamente

\newcommand{\sigla}[3]{\item[#2] #3
\acrodef{#1}[#2]{#3}\index{#2}}

\begin{siglas}
  \sigla{abnt}{ABNT}{Associação Brasileira de Normas Técnicas}
  \sigla{abntex}{abnTeX}{ABsurdas Normas para TeX}
  \sigla{faq}{FAQ}{\emph{Frequently asked questions} - Perguntas frequentes}
  \sigla{ifsp}{IFSP}{Instituto Federal de Educação, Ciência e Tecnologia de São Paulo}
\end{siglas}



% ---

% \input{pre-simbolos}


% ---
% inserir o sumario
% ---
\pdfbookmark[0]{\contentsname}{toc}
\tableofcontents*
\cleardoublepage
% ---


% ----------------------------------------------------------
% ELEMENTOS TEXTUAIS
% ----------------------------------------------------------
\textual

% Para facilitar a manutenção é sempre melhore criar um arquivo por capitulo, para exemplo isso não é necessário
\chapter{Resumo}
	A ideia deste experimento é implementar uma maquina de estado utilizando a linguagem Verilog.
	A maquina tenta representar uma situação do mundo real de um portão de garagem.

	\begin{quote}
		Considere o cenário de um controle para um portão de garagem. Em um estado inicial, o portão
		está fechado. Caso um acionador externo seja selecionado, o portão abre. Caso o portão esteja
		aberto e o acionador externo seja selecionado, o portão fecha. O portão nunca abre e fecha ao
		mesmo tempo. O trilho no qual o portão se desloca é equipado com dois sensores que indicam
		quando o portão está completamente aberto e quando está completamente fechado. O motor
		não deve tentar abrir o portão quando esse estiver aberto e nem deve fechá-lo quando este já
		estiver fechado.

		Para maior segurança dos usuários, o motor está equipado com um aviso luminoso que deve ser
		acesso quando o portão se desloca.

		Deve-se assumir que não é possível parar o portão enquanto ele estiver abrindo ou fechando,
		mas é possível que o usuário aperte o acionador externo enquanto o portão estiver se
		deslocando. Nesse caso, se o portão estiver abrindo, ele deve passar a fechar e vice-versa.
	\end{quote}

%Apresentar  o  objetivo  do  experimento e  sua  relação  com  o  mundo  real.
%Citar exemplos  de  produtos  e  dados  de  empresas  que  usam  circuitos  semelhantes.
%Dados  históricos  e  estatísticos,  em  temas  próximos,  são  desejáveis.
%Adicionar referências bibliográficas.

% Para facilitar a manutenção é sempre melhore criar um arquivo por capitulo, para exemplo isso não é necessário

%---------------------------------------------------------------------------------------
\chapter{Descrição da execução do experimento}

\section{Cenario 1}

	Foram necessários para o desenvolvimento do experimento:
	\begin{itemize}
		\item Multímetro Digital
		\item \ac{ci} de portas lógicas \textit{AND} ( \textit{datasheet} 7400 )
		\item \ac{ci} de portas lórigas \textit{OR}
		\item \ac{ci} de porta lógica inversora / \textit{NOT}( \textit{datashee} 7404)
		\item \textit{Protoboard}
		\item Fios para conectar as portas
		\item Fonte de Alimentação DC 5V
		\item LED Vermelho
		\item LED Verde
		\item 2 resistores para polarizar os LED’s
		\item Alicate
	\end{itemize}

	A partir do problema proposto, montou-se a seguinte expressão lógica
	$$ P . ( F + C + O) + (F . C . O)$$

	com P representando \textit{o voto do presidente}, F
	\textit{o voto do diretor financeiro}, C
	\textit{o voto do controller} e O \textit{o voto do diretor de operações}, após a
	montagem da expressão, foi elaborada a \autoref{table:tabelaVerdade1}. Com esta tabela e a expressão lógica,
	elaborou-se o circuito, conforme a \autoref{fig:desenhoCircuito1}. Com tais informações, foi repassado o circuito
	para o Quartus, depois renomeou-se as entradas e saídas para que, por meio do arquivo tradutor, a placa
	\ac{fpga} reconhecesse os componentes, conforme \autoref{fig:printCircuito1}.
	Para cobrir todos os casos de testes, foi realizada uma simulação, conforme a \autoref{fig:printSimulacoes}.

	\begin{table}[h]
		\centering
		\caption{Tabela verdade da expressão lógica do cenário 1}\label{table:tabelaVerdade1}
		\begin{tabular}{c|c|c|c|c}
		%\hline
			\textbf{P} & \textbf{F} & \textbf{C} & \textbf{O} & \textbf{P.(F+C+O)+(F.C.O)} \\
			\hline
			0 & 0 & 0 & 0 & 0\\\hline
			0 & 0 & 0 & 1 & 0\\\hline
			0 & 0 & 1 & 0 & 0\\\hline
			0 & 0 & 1 & 1 & 0\\\hline
			0 & 1 & 0 & 0 & 0\\\hline
			0 & 1 & 0 & 1 & 0\\\hline
			0 & 1 & 1 & 0 & 0\\\hline
			0 & 1 & 1 & 1 & 1\\\hline
			1 & 0 & 0 & 0 & 1\\\hline
			1 & 0 & 0 & 1 & 1\\\hline
			1 & 0 & 1 & 0 & 1\\\hline
			1 & 0 & 1 & 1 & 1\\\hline
			1 & 1 & 0 & 0 & 1\\\hline
			1 & 1 & 0 & 1 & 1\\\hline
			1 & 1 & 1 & 0 & 1\\\hline
			1 & 1 & 1 & 1 & 1\\

		\end{tabular}
	\end{table}

	\begin{figure}[H]
		\centering
		\caption{\label{fig:desenhoCircuito1}Desenho do circuito do cenário 1}
		\includegraphics[width=1\textwidth]{img/cenario1/desenhoCircuito}
	\end{figure}

	\begin{figure}[H]
		\centering
		\caption{\label{fig:printCircuito1}Imagem do circuito no Quartus do cenário 1}
		\includegraphics[width=1\textwidth]{img/cenario1/printCircuito}
	\end{figure}

	A porta SW[9] representa P, a SW[8] representa F, a SW[7]
	 representa C, SW[6] representa O, LEDR[1] é um led vermelho que indica
	 que o resultado da votação foi falso e LEDG[1] é um led verde que representa o resultado da votação
	 foi verdadeiro.

	\begin{figure}[H]
		\centering
		\caption{\label{fig:protoboard1}Configuração onde o LED Verde deveria acender (1001, por exemplo)}
		\includegraphics[width=1\textwidth]{img/cenario1/protoboard1}
	\end{figure}

	\begin{figure}[H]
		\centering
		\caption{\label{fig:protoboard2}Configuração onde o LED Vermelho deveria acender (0001, por exemplo)}
		\includegraphics[width=1\textwidth]{img/cenario1/protoboard2}
	\end{figure}

\section{Cenario 2}
	Para a realização deste experimento, foram utilizados o programa Quartus 13.0 SP 1 e a placa \ac{fpga}
	Cyclone II - EP2C20F484C7.

	A partir do problema proposto, montou-se a seguinte expressão lógica
	$$ P + G + \sim V$$
	com P representando \textit{se a porta estiver aberta}, G
	\textit{se nível de gelo do congelador estiver acima do permitido} e V
	\textit{se o nível de gás do motor estiver adequado}, após a
	montagem da expressão, foi elaborada a \autoref{table:tabelaVerdade2}. Com esta tabela e a expressão lógica,
	elaborou-se o circuito, conforme a \autoref{fig:desenhoCircuito2}. Com tais informações, foi repassado o circuito
	para o Quartus, depois renomeou-se as entradas e saídas para que, por meio do arquivo tradutor, a placa
	\ac{fpga} reconhecesse os componentes.
	Para cobrir todos os casos de testes, foi realizada uma simulação, conforme a \autoref{fig:printSimulacao}.

	\begin{table}[h]
		\centering
		\caption{Tabela verdade da expressão lógica do cenário 2}\label{table:tabelaVerdade2}
		\begin{tabular}{c|c|c|c}
		%\hline
			\textbf{P} & \textbf{G} & \textbf{$\sim$V} & \textbf{P+G+($\sim$V)} \\
			\hline
			0 & 0 & 0 & 0\\\hline
			0 & 0 & 1 & 1\\\hline
			0 & 1 & 0 & 1\\\hline
			0 & 1 & 1 & 1\\\hline
			1 & 0 & 0 & 1\\\hline
			1 & 0 & 1 & 1\\\hline
			1 & 1 & 0 & 1\\\hline
			1 & 1 & 1 & 1
		\end{tabular}
	\end{table}

	\begin{figure}[H]
	    \centering
		\caption{\label{fig:desenhoCircuito2}Desenho do circuito do cenário 2}
		\includegraphics{img/cenario2/desenhoCircuito}
	\end{figure}


	\begin{figure}[H]
	    \centering
		\caption{\label{fig:printCircuito2}Imagem do circuinto no programa Quartus do cenário 2}
		\includegraphics[width=1\textwidth]{img/cenario2/printCircuito}
	\end{figure}

	A porta SW[9] representa a P, a SW[8] representa a G, a SW[7]
	 representa a $\sim$V e a LEDR[1] é um led vermelho que irá
	 indicar o resultado provido da expressão lógica. Uma observação que não merece uma devida atenção é que
	 na \autoref{fig:desenhoCircuito2} foram necessárias a utilização de duas portas \textit{OR}, enquanto na
	 \autoref{fig:printCircuito2} foi necessária apenas a utilização de uma porta \textit{OR}. Isso ocorreu pelo fato
	 de que no Quartus existe a possibilidade de utilizar uma porta \textit{OR} de três entradas.

	Por fim, o circuito virtual foi compilado, conforme \autoref{fig:printCompilacao2}.

	\begin{figure}[H]
	    \centering
		\caption{\label{fig:printCompilacao2}Resultado da compilação do circuito do cenário 2}
		\includegraphics[width=1\textwidth]{img/cenario2/printCompilacao}
	\end{figure}


%Apresentar   o   detalhamento   da  execução  e   resultados   dos   passos   realizados
%durante   o   experimento,   incluindo   tabelas   verdade,   esquemáticos,   e   código
%(quando  houver).
%Especificar  componentes,  sistemas  e  instrumentos  utilizados.
%Usar listas, figuras e quadros, descrevê-los e discuti-los.

\chapter{Avaliação dos resultados do experimento}
	A entrada formada por ( b, a, f, m, s) é correspondida respectivamente pelas entradas
	SW[4], SW[3], SW[2], SW[1] e SW[0]. Para o estado Abrindo acende o \ac{led} verde (LEDG[0]) e
	para o estado de Fechando acende o \ac{led} vermelho (LEDR[0]). Assim, ao executar o
	\textit{test bench} obteve-se os resultados conforme as
	\autoref{figura:testBenchWaveMaquina} e \autoref{figura:testBenchTranscriptMaquina}.

	\begin{figure}[H]
		 \centering
		 \caption{\label{figura:testBenchWaveMaquina}\textit{Test bench Wave} do código da máquina.}
		 \includegraphics[width=1\textwidth]{img/maquina/testBenchWave}
	\end{figure}

	\begin{figure}[H]
		 \centering
		 \caption{\label{figura:testBenchTranscriptMaquina}\textit{Test bench Transcript} do código da máquina.}
		 \includegraphics[width=0.5\textwidth]{img/maquina/testBenchTranscript}
	\end{figure}

	Como resultado do \textit{deploy} na placa, obteve o resultado confome as
	\autoref{figura:deployMaquina1}, \autoref{figura:deployMaquina2}, \autoref{figura:deployMaquina3},
	\autoref{figura:deployMaquina4}, \autoref{figura:deployMaquina5}, \autoref{figura:deployMaquina6} e
	\autoref{figura:deployMaquina7}.

		\begin{figure}[H]
			\includegraphics[width=1\textwidth]{img/maquina/placa/Fechado}
			\caption{Máquina no estado inicial (Fechado).\label{figura:deployMaquina1}}
		\end{figure}

		\begin{figure}[H]
			\includegraphics[width=1\textwidth]{img/maquina/placa/Fechado-Abrindo}
			\caption{Transição do estado Fechado para o Abrindo.\label{figura:deployMaquina2}}
		\end{figure}

		\begin{figure}[H]
			\includegraphics[width=1\textwidth]{img/maquina/placa/Abrindo-Aberto}
			\caption{Transição do estado Abrindo para o Aberto.\label{figura:deployMaquina3}}
		\end{figure}

		\begin{figure}[H]
			\includegraphics[width=1\textwidth]{img/maquina/placa/Abrindo-Fechando}
			\caption{Transição do estado Abrindo para o Fechando.\label{figura:deployMaquina4}}
		\end{figure}

		\begin{figure}[H]
			\includegraphics[width=1\textwidth]{img/maquina/placa/Aberto-Fechando}
			\caption{Transição do estado Aberto para o Fechando.\label{figura:deployMaquina5}}
		\end{figure}

		\begin{figure}[H]
			\includegraphics[width=1\textwidth]{img/maquina/placa/Fechando-Abrindo}
			\caption{Transição do estado Fechando para Abrindo.\label{figura:deployMaquina6}}
		\end{figure}

		\begin{figure}[H]
			\includegraphics[width=1\textwidth]{img/maquina/placa/Fechando-Fechado}
			\caption{Transição do estado Fechando para Fechado.\label{figura:deployMaquina7}}
		\end{figure}


%Apresentar os resultados da simulação em software e da utilização do Kit DE1 e/ou
%protoboard. Utilizar figuras, descrevê-las e discuti-las.

\chapter{Análise crítica e discussão}
	\section{ETAPA 1 – Display de 7 segmentos}
		Teve-se dificuldade para entender que era necessário criar um circuito
		para para cada segmento do display e na leitura do resultado da simulação.
	\section{ETAPA 2 – Meio-somador 1 bit}
		Teve-se dificuldade de como implementar o meio-somador utilizando apenas as portas NAND.
	\section{ETAPA 3 – Meio-somador 4 bits}
		Teve-se dificuldade para dividir o resultado em dois \textit{display}.


%Apresentar  a  visão do  grupo  sobre  o  experimento,  apresentando  pontos  fáceis  e
%de  dificuldades  para  a  realização  do  mesmo.  Comente  se  os  resultados  obtidos
%representam  o  comportamento  esperado   do   grupo   para  o   circuito,   fazendo
%relação com o conteúdo teórico.

\chapter{Outras informações}
	Para estudar como passar a ideia da máquina de estado para um código em
	Verilog, pensou-se numa máquina simples, que tem como estado inicial o estado A,
	que vai para um estado B e o B vai para o A. Assim, para determinar em que estado
	a máquina encontra-se, estipulou-se \ac{led}s para representar o estado atual,
	\ac{led} verde para o estado A e para B um \ac{led} vermelho.

	O desenho da máquina encontra-se na \autoref{figura:adicional},
	o Verilog no \autoref{codigo:adicional}, a compilação na
	\autoref{figura:compilacaoAdicional}, código do teste
	no \autoref{codigo:testBenchAdicional} e os resultados dos testes nas
	\autoref{figura:testBenchWaveAdicional} e \autoref{figura:testBenchTranscriptAdicional}.

	\begin{figure}[H]
		\includegraphics{img/maquinaAdicional/maquinaEstado}
		\caption{Máquina de estado do circuito adicional.}
		\label{figura:adicional}
	\end{figure}

	\lstinputlisting[label = {codigo:adicional},caption = {Código da máquina de estado adicional.}]
	{codigo/circuitoAdicional.v}

	\begin{figure}[H]
		\centering
		\includegraphics[width=0.4\textwidth]{img/maquinaAdicional/compilacao}
		\caption{Compilação da máquina de estado do circuito adicional.}
		\label{figura:compilacaoAdicional}
	\end{figure}

	\lstinputlisting[label = {codigo:testBenchAdicional},caption = {Código do \textit{test bench} da máquina de estado adicional.}]
	{codigo/circuitoAdicional_TB.v}

	\begin{figure}[H]
		\includegraphics[width=1\textwidth]{img/maquinaAdicional/testBenchWave}
		\caption{\textit{Test bench Wave} da máquina de estado adicional.}
		\label{figura:testBenchWaveAdicional}
	\end{figure}

	\begin{figure}[H]
		\centering
		\includegraphics{img/maquinaAdicional/testBenchTranscript}
		\caption{\textit{Test bench Transcript} da máquina de estado adicional.}
		\label{figura:testBenchTranscriptAdicional}
	\end{figure}

% exemplos de escrita LaTeX
	%\input{exemplos/exemplos}
	%\input{textos-conclusao}

% ----------------------------------------------------------
% Finaliza a parte no bookmark do PDF
% para que se inicie o bookmark na raiz
% e adiciona espaço de parte no Sumário
% ----------------------------------------------------------
\phantompart

% ----------------------------------------------------------
% ELEMENTOS PÓS-TEXTUAIS
% ----------------------------------------------------------
\postextual
% ----------------------------------------------------------

% ----------------------------------------------------------
% Referências bibliográficas
% ----------------------------------------------------------
	%\bibliography{referencias,exemplos/abntex2-doc-abnt-6023}

%\input{pos-glossario.tex}

% ----------------------------------------------------------
% Apêndices
% ----------------------------------------------------------

% ---
% Inicia os apêndices
% ---
\begin{apendicesenv}

% Imprime uma página indicando o início dos apêndices
\partapendices

\chapter{Testbench meio-somador2}
	\label{apendice:meio_somador2}
	\lstinputlisting[caption={Testbench do módulo meio-somador2.}]{img/meio_somador2_TB.v}

\chapter{Testbench meio-somador3}
	\label{apendice:meio_somador3}
	\lstinputlisting[caption={Testbench do módulo meio-somador3.}]{img/meio_somador3_TB.v}

\chapter{Testbench somador completo 4 bits}
	\label{apendice:somador-completo}
	\lstinputlisting[caption={Testbench do módulo de somador completo 4 bits.}]{img/Somador4Bits_TB.v}



\end{apendicesenv}
% ---

% ----------------------------------------------------------
% Anexos
% ----------------------------------------------------------

% ---
% Inicia os anexos
% ---
\begin{anexosenv}

% Imprime uma página indicando o início dos anexos
\partanexos

% ---
\chapter{\textit{Datasheet} do componente 7449}
% ---
\index{pdf}
% somente algumas páginas
\label{anexo:datasheet-7449}
\includepdf[pages=1-4,frame=false]{anexos/7449.pdf}

\end{anexosenv}



%---------------------------------------------------------------------
% INDICE REMISSIVO - Quando necessário
% As palavras indexadas devem ser definidas com \index{} no texto
%---------------------------------------------------------------------
\phantompart
	%\printindex

%---------------------------------------------------------------------

\end{document}
