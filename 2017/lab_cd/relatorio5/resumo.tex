\chapter{Resumo}
	A ideia deste experimento é implementar uma máquina de estado utilizando a linguagem de descrição
	de \textit{hardware} Verilog.
	A máquina tenta representar uma situação do mundo real de um portão de garagem.

	\begin{quote}
		Considere o cenário de um controle para um portão de garagem. Em um estado inicial, o portão
		está fechado. Caso um acionador externo seja selecionado, o portão abre. Caso o portão esteja
		aberto e o acionador externo seja selecionado, o portão fecha. O portão nunca abre e fecha ao
		mesmo tempo. O trilho no qual o portão se desloca é equipado com dois sensores que indicam
		quando o portão está completamente aberto e quando está completamente fechado. O motor
		não deve tentar abrir o portão quando esse estiver aberto e nem deve fechá-lo quando este já
		estiver fechado.

		Para maior segurança dos usuários, o motor está equipado com um aviso luminoso que deve ser
		acesso quando o portão se desloca.

		Deve-se assumir que não é possível parar o portão enquanto ele estiver abrindo ou fechando,
		mas é possível que o usuário aperte o acionador externo enquanto o portão estiver se
		deslocando. Nesse caso, se o portão estiver abrindo, ele deve passar a fechar e vice-versa.
	\end{quote}

	Para solucionar tal problema e facilitar sua resolução, dividiu-se o processo em quatro passos:
	\begin{enumerate}
	   \item Desenhar a máquina de estado para o cenário em questão;
	   \item Escrever um código Verilog para a máquina de estado no passo anterior;
	   \item Executar o código na \ac{fpga} e simulação.
	 \end{enumerate}

	 Além disso, escolheu-se uma máquina de estado qualquer para estudar como
	 implementá-la em Verilog.
%Apresentar  o  objetivo  do  experimento e  sua  relação  com  o  mundo  real.
%Citar exemplos  de  produtos  e  dados  de  empresas  que  usam  circuitos  semelhantes.
%Dados  históricos  e  estatísticos,  em  temas  próximos,  são  desejáveis.
%Adicionar referências bibliográficas.
