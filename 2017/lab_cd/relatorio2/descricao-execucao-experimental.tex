% Para facilitar a manutenção é sempre melhore criar um arquivo por capitulo, para exemplo isso não é necessário

%---------------------------------------------------------------------------------------
\chapter{Descrição da execução do experimento}

	Para a realização deste experimento, foram utilizados o programa Quartus 13.0 SP 1 e a placa \ac{fpga}
Cyclone II - EP2C20F484C7.

	\section{ETAPA 1 – Display de 7 segmentos}
		Para representar um número de 4 \textit{bits} na placa, utilizou-se 4 \textit{switch}, cada um
		representando um bit do número. Como um segmento do \textit{display} poderia ser acendido
		em mais de um número,
		motou-se uma expressão lógica para cada segmento do \textit{display}.

		Para o display 0 montou-se a expressão
		$$D’.C’.A’ + A.C.B + A.C.D’ + D’.B$$,
		para o display 1 montou-se a expressão
		$$D’.C+D’.A’.B’+A.B.C$$,
		para o display 2 montou-se a expressão
		$$C.D’+A.C’.D’+B’.C’.D’+A.B.C$$,
		para o display 3 montou-se a expressão
		$$C’.D’.A+C’.D’.B+A’.B.D’+A.B.C.D+A.B’C.D’$$,
		para o display 4 montou-se a expressão
		$$A’.C’.D’+A.B.C.D+A.B’.C.D’$$,
		para o display 5 montou-se a expressão
		$$A’.B’.D’+A’.C.D’+A.B.C+A.C.D’$$,
		para o display 6 montou-se a expressão
		$$C.D’+B.C’.D’$$.

		Com tais expressões, montamos o circuito conforme o \autoref{apendice:CircuitoEtapa1}.

	\section{ETAPA 2 – Meio-somador 1 bit}
		
	\section{ETAPA 3 – Meio-somador 4 bits}





%Apresentar   o   detalhamento   da  execução  e   resultados   dos   passos   realizados
%durante   o   experimento,   incluindo   tabelas   verdade,   esquemáticos,   e   código
%(quando  houver).
%Especificar  componentes,  sistemas  e  instrumentos  utilizados.
%Usar listas, figuras e quadros, descrevê-los e discuti-los.



%---------------------------------------------------------------------------------------
