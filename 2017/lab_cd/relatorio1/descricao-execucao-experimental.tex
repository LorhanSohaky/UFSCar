% Para facilitar a manutenção é sempre melhore criar um arquivo por capitulo, para exemplo isso não é necessário

%---------------------------------------------------------------------------------------
\chapter{Descrição da execução do experimento}
Para a realização deste experimento, foram utilizados o programa Quartus 13.0 SP 1 e a placa \ac{fpga} Cyclone II - EP2C20F484C7.

A partir do problema proposto, montou-se a seguinte expressão lógica
$$ P + G + \sim V$$
com P representando \textit{porta aberta}, G \textit{gelo} e V \textit{vazamento de gás}, após a
montagem da expressão, foi elaborada a \autoref{table:tabelaVerdade2}. Com esta tabela e a expressão lógica,
elaborousse o circuito, conforme a \autoref{fig:desenhoCircuito2}. Com tais informações, foi repassado o circuito
para o Quartus, depois renomeou-se as entradas e saídas para que, por meio do arquivo tradutor, a placa
\ac{fpga} reconhecesse os componentes.
Para cobrir todos os casos de testes, foi realizada uma simulação, conforme a \autoref{fig:printSimulacao}.

\begin{table}[h]
	\centering
	\caption{Tabela verdade da expressão lógica}\label{table:tabelaVerdade2}
	\begin{tabular}{c|c|c|c}
	%\hline
		\textbf{P} & \textbf{G} & \textbf{V} & \textbf{P+G+($\sim$V)} \\
		\hline
		0 & 0 & 0 & 0\\\hline
		0 & 0 & 1 & 1\\\hline
		0 & 1 & 0 & 1\\\hline
		0 & 1 & 1 & 1\\\hline
		1 & 0 & 0 & 1\\\hline
		1 & 0 & 1 & 1\\\hline
		1 & 1 & 0 & 1\\\hline
		1 & 1 & 1 & 1
	\end{tabular}
\end{table}

\begin{figure}[htb]
    \centering
	\caption{\label{fig:desenhoCircuito2}Desenho do circuito}
	\includegraphics{\ifspprefixo/LogoUFSCar.jpg}
\end{figure}


Apresentar   o   detalhamento   da  execução  e   resultados   dos   passos   realizados
durante   o   experimento,   incluindo   tabelas   verdade,   esquemáticos,   e   código
(quando  houver).
Especificar  componentes,  sistemas  e  instrumentos  utilizados.
Usar listas, figuras e quadros, descrevê-los e discuti-los.



%---------------------------------------------------------------------------------------
