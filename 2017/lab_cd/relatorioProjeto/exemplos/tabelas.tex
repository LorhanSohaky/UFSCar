\subsection{Tabelas e Quadros}
A seção 3.32 da NBR14724:2011 define a Tabela como sendo uma "forma não discursiva de apresentar informações das quais o dado numérico se destaca como informação central" 

Quadros e tabelas são informações tabulares, mas Tabelas tem como objetivo apresentar números.

Uso de tabelas no \LaTeX : \href{url}{https://en.wikibooks.org/wiki/LaTeX/Tables}


\index{quadros}O \autoref{quadro-exemplo} é um exemplo de dados tabulares gerados em 
\LaTeX.

\begin{quadro}[htb]
\centering
\ABNTEXfontereduzida
\caption[Níveis de investigação]{Níveis de investigação.}
\label{quadro-exemplo}
\begin{tabular}{|p{2.6cm}|p{6.0cm}|p{2.25cm}|p{3.40cm}|}
  \hline
   \textbf{Nível de Investigação} & \textbf{Insumos}  & \textbf{Sistemas de Investigação}  & \textbf{Produtos}  \\
    \hline
    Meta-nível & Filosofia\index{filosofia} da Ciência  & Epistemologia &
    Paradigma  \\
    \hline
    Nível do objeto & Paradigmas do metanível e evidências do nível inferior &
    Ciência  & Teorias e modelos \\
    \hline
    Nível inferior & Modelos e métodos do nível do objeto e problemas do nível inferior & Prática & Solução de problemas  \\
   \hline
\end{tabular}
\legend{Fonte: Próprio Autor}
\end{quadro}



\index{tabelas}Já a \autoref{tab-exemplo} foi criada conforme o padrão do IBGE
requerido pelas normas da ABNT para documentos técnicos e acadêmicos. Números devem ser alinhados a direita.

\begin{table}[htb]
\centering
\caption{Um Exemplo de tabela}
\label{tab-exemplo}
\begin{tabular}{p{2.6cm}|r|r|r}
    \hline
   \textbf{Item} & \textbf{Janeiro}  & \textbf{Fevereiro}  & \textbf{Março}  \\
    \hline
    Classes & 2  & 10 & 20  \\
    \hline
    Linhas & 100  & 250 & 543 \\
    \hline
\end{tabular}
\fonte{Dados do Projeto}
\end{table}

\def\equationautorefname~#1\null{%
  Equação~(#1)\null
}


\index{equação}\index{Pitagoras}A \autoref{eq-pythagoras} mostra que também é possivel escrever equações diretamente em \LaTeX

\begin{equation}\label{eq-pythagoras}
a^2+b^2=c^2\,.
\end{equation}





